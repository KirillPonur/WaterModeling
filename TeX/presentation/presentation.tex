\documentclass[10pt,pdf,hyperref={unicode}, dvipsnames]{beamer}
%!TEX root = ../plasma.tex
\usepackage[english,russian]{babel}
% \usepackage[T2A,T1]{fontenc}
\usepackage[utf8]{inputenc}
% \usepackage{tikz}
\usepackage[unicode]{hyperref}
% \usepackage{pgfplots,standalone}
\usepackage{caption}
\usepackage[normalem]{ulem}
\usepackage
	{
		% Дополнения Американского математического общества (AMS)
		amssymb,
		amsfonts,
		amsmath,
		amsthm,
		physics
		}
% \usepackage{lmodern}
% \pgfplotsset{compat=newest} 
% \usetikzlibrary{%
%     decorations.pathreplacing,%
%     decorations.pathmorphing,%
%     patterns,%
%     angles,%
%     quotes,%
%     calc, %
%     3d, %
%     backgrounds, %
%     positioning%
% }


% Стиль презентации

 \usetheme{default}
 \usefonttheme{professionalfonts}
 \usecolortheme{}
 % \usecolortheme{whale}
% \let\oldframe\enumerate
% \renewcommand{\frame}{%
% \oldframe
% \let\olditemize\itemize
% \renewcommand\itemize{\olditemize\addtolength{\itemsep}{100pt}}%
% }
 

% \setbeamercolor{frametitle right}{fg=white,bg=Brown!85}
% \setbeamercolor{frametitle}{fg=white,bg=Brown!85}
%\setbeamercolor{frametitle right}{fg=white,bg=black!85} %
\setbeamercolor{frametitle}{fg=white,bg=black!85} % Цвет титульника
\setbeamercolor{item projected}{fg=white,bg=black!85} % Цвет титульника
\setbeamertemplate{blocks}[rounded][shadow=false] %стиль блоков
\setbeamertemplate{itemize item}{\color{black!85}$\bullet$}
\setbeamertemplate{headline}{}
\setbeamertemplate{footline}{} 
\setbeamertemplate{navigation symbols}{} % минус навигация
% \let\Tiny=\tiny % решает проблему со шрифтами в TexLive
%\setbeamertemplate
%	{footline}{
%		\color{black!40!white}
%		\quad\hfill
%		\insertframenumber/\inserttotalframenumber
%		\hfill\vspace{1cm}\quad
%	} 


\beamersetrightmargin{1cm} 
\beamersetleftmargin{1cm}

\setbeamertemplate{enumerate item}
{
	\usebeamercolor[bg]{item projected}
	\raisebox{1pt}{\colorbox{black!85}{\color{fg}\footnotesize\insertenumlabel}}%
}

% \setbeamertemplate{itemize item}{%
% 	\usebeamercolor[bg]{item projected}%
% 	\raisebox{3pt}{{\colorbox{black!85}\footnotesize$\bf$\bullet}}%
% }

\setbeamercolor{item projected}{bg=black,fg=white}
\setbeamercolor{title}{bg=black!85,fg=white}

\setbeamertemplate{frametitle}
{	
	\nointerlineskip
	\begin{beamercolorbox}[sep=15pt,ht=1.9em,wd=\paperwidth]{frametitle}
		% \vspace{}%
		\strut\insertframetitle\strut
		\vskip-1.8ex%	
	\end{beamercolorbox}
}


\renewcommand{\phi}{\varphi}
\renewcommand{\epsilon}{\varepsilon}
\renewcommand{\div}{\operatorname{div}}
\begin{document}
\title[Измерение плотности плазмы]{Численное моделирование морской поверхности}

\author{%
	Понур К.А. %
}

\institute{Национальный исследовательский Нижегородский государственный университет имени Н. И. Лобачевского, \\ Радиофизический факультет}

%!TEX root = ../plasma.tex
\begin{frame}[plain]
	
	\begin{center}
		\small{\insertinstitute}
		\vspace{1cm}
	\end{center}
		\begin{beamercolorbox}[sep=8pt,center]{title}
			\usebeamerfont{title}\inserttitle
		\end{beamercolorbox}
		\vspace{0.1cm}
	\begin{flushright}
		\normalsize \textbf{Работу выполнил:}\\
		\large
		\insertauthor \\
		\vspace{0.5cm}
		\normalsize{\textbf{Научный руководитель:}\\}
		\large{Караев В.Ю.}
		\vfill
	\end{flushright}

	\centering{\small{\today }}
\end{frame}

\section{Введение}
\subsection{Цели работы}
\begin{frame}[t]
	\frametitle{Цели работы}
	% \textbf{Цели}\\
		\vfill
		\begin{enumerate}
			% \item \sout{Получить зачёт по УНЭ.}
			\item Изучить принципы моделирования морской поверхности.

			\item Оптимизировать существующие алгоритмы.


		\end{enumerate}
		\vfill
\end{frame}
\subsection{Актуальность работы}
\begin{frame}[t]

	\frametitle{Актуальность работы}

\end{frame}


\begin{frame}
	\frametitle{Основные понятия}

\end{frame}

\begin{frame}[t]

	\frametitle{Двумерная модель}
    
\end{frame}


\begin{frame}[t]

	\frametitle{Реальные и модельные поля}
    
\end{frame}

\begin{frame}[t]
	
	\frametitle{<<Отбеливание>> спектра}
   
\end{frame}

\begin{frame}[t]

	\frametitle{Модель поверхностного волнения}
    
\end{frame}

\end{document}